\documentclass[]{iat}

%Hier können weitere benötigte Pakete Eingebunden werden
%\usepackage[backend=biber,style=alphabetic,]{biblatex}
\usepackage{float}
\usepackage{listingsutf8}
\usepackage{color}
\usepackage{gensymb}
\usepackage{caption}
\usepackage{hyperref}
%Namen eingeben / Insert Name
\renewcommand{\author}{Nico Spiske}
%Art der Arbeit / Scope (Project, Thesis...)
\providecommand{\scope}{Bachelorarbeit}
%Thema der Arbeit / Theme of Thesis
\renewcommand{\subject}{Lösungsansätze für die Regelung von dynamischen Systemen mittels
Reinforcement Learning Algorithmen}
%Schlagwörter / Keywords
\providecommand{\keywords}{Fernwärmenetz, Projektarbeit, Projekt, Maschinelles Lernen}
%Literaturliste *.bib / Bibliography
\providecommand{\bibfile}{literature}
%Matrikelnummer / Student ID
\providecommand{\studentID}{4436923}
%Betreuer /& Tutors
\providecommand{\tutora}{M.Sc. Ricardo Bosold}
\providecommand{\tutorb}{M.Sc. Phillipp Hendrys}
%Prüfer / Examiner
\providecommand{\examinera}{Prof. Dr.-Ing. Kai Michels}
\providecommand{\examinerb}{Dr.-Ing. Dennis Pierl}
%Farben
\definecolor{dkgreen}{rgb}{0,0.6,0}
\definecolor{gray}{rgb}{0.5,0.5,0.5}
\definecolor{mauve}{rgb}{0.58,0,0.82}
\definecolor{darkblue}{rgb}{0.0,0.0,0.6}
\definecolor{cyan}{rgb}{0.0,0.6,0.6}
%lst überschrift
\captionsetup[lstlisting]{font={Large}}

\hypersetup{%
	pdftitle	={\subject -- \author -- \today},
	pdfauthor	={\author},
	pdfsubject	={\subject},
	pdfkeywords ={\keywords}
}

\addbibresource{literature.bib}

\setlength{\footheight}{21pt}

\begin{document}
%Pfad zu Grafiken:
\graphicspath{{./project_graphics/}}
% Sprachauswahl /Language Selection (ngerman/english)
\selectlanguage{ngerman}
% \pagenumbering{roman}
\begin{titlepage}
    \newgeometry{left=0cm,right=0cm,top=0cm,bottom=0cm}
    {\color{iatred}\rule{\textwidth}{1cm}}
    \begin{minipage}{0.7\textwidth}
        \vspace{-1.37cm}
        {\color{iatred}\rule{\textwidth}{0.5cm}}
    \end{minipage}%
    \begin{minipage}{0.3\textwidth}
        \raggedright
        \vspace{0.4cm}
        \hspace{0.35cm}
        \includegraphics[scale=1]{Logos/iat_logo_en}
    \end{minipage}%
    \centering
    \vspace{4cm}
    
        {\Large
            \textbf{\scope}
        }\par
    
    \vspace{2cm}
    
        {\linespread{1.1}\huge\sffamily\bfseries
         \subject\par}
    
    \vspace{2cm}
    
        {\large
         \author\\
         \studentID
        \par}
    
    \vspace{1cm}
    
        \today
        
    \vspace{3cm}
    \begin{table}[h]
        \centering
        \begin{tabular}{lll}
            \underline{Betreuer:}	&\quad\hspace{4cm}\quad&\underline{Gutachter:}\hspace{5cm}\\
            \tutora	&	&\examinera\\
            \tutorb	&	&\examinerb
        \end{tabular}
    \end{table}
    \thispagestyle{empty}  
    \vfill
    \begin{minipage}{0.3\textwidth}
        \centering
    %	\hspace{2cm}
        \includegraphics[scale=1.2]{Logos/uni_logo_title}\relax
            \vspace{0.1cm}
    \end{minipage}%
    \begin{minipage}{0.7\textwidth}
        \raggedleft
        {\color{iatred}\rule{\textwidth}{0.5cm}}
        \vspace{-0.9cm}
    \end{minipage}
    {\color{iatred}\rule{\textwidth}{1cm}}
    \end{titlepage}
    
%Urherberrechtserklärung / Confirmation of Conformity Comment if not needed
\chapter*{Urheberrechtliche Erklärung}
Hiermit versichere ich, dass ich meine Abschlussarbeit ohne fremde Hilfe angefertigt habe und dass ich keine Anderen als die von mir angegebenen Quellen und Hilfsmittel benutzt habe.\par
Alle Stellen, die wörtlich oder sinngemä{\ss} aus Veröffentlichungen entommen sind, habe ich unter Angabe der Quellen als solche kenntlich gemacht.\par
Die Abschlussarbeit darf nach der Abgabe nicht mehr verändert werden.\par
\vspace{2.5em}
Datum:\underline{\hspace{3.5cm}}\qquad Unterschrift:\underline{\hspace{5.5cm}}
\vspace{3em}
\section*{Erklärung zur Veröffentlichung von Abschlussarbeiten}
$\Box$ Ich bin damit einverstanden, dass meine Abschlussarbeit im Universitätsarchiv für wissenschaftliche Zwecke von Dritten eingesehen werden darf.\par
$\Box$ Ich bin damit einverstanden, dass meine Abschlussarbeit nach 30 Jahren (gem. §7 Abs.2 BremArchivG) im Universitätsarchiv f`ür wissenschaftliche Zwecke von Dritten eingesehen werden darf.\par
$\Box$ Ich bin \textit{nicht} damit einverstanden, dass meine Abschlussarbeit im Universitätsarchiv für wissenschaftliche Zwecke von Dritten eingesehen werden darf.\par
\vspace{2.5em}
Datum:\underline{\hspace{3.5cm}}\qquad Unterschrift:\underline{\hspace{5.5cm}}


\tableofcontents

\newpage

\chapter{Einleitung}
\section{Einleitung und Motivation}
\section{Zielsetzung}
\section{Aufbau der Arbeit}

\chapter{Grundlagen des Deep Learnings}
\section{Was ist Deep Learning?}
\section{Grundkonzepte und Terminologie}
\section{Artes des Deep Learnings}
\section{Soft Actor-Critic (SAC) Agenten}

\chapter{Methode und Umsetzung}
\section{Verwendete Software und Bibliotheken}
\section{Implementierung des Agenten}
\section{Umsetzung des Modells in Simulink}
\section{Einbindung des Modells in Python}
f
\chapter{Ergebnisse}

\chapter{Diskussion}

\chapter{Zusammenfassung und Ausblick}

\end{document}